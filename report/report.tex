\documentclass[a4paper,12pt]{article}

\usepackage{a4wide}
\usepackage{amsmath}
\usepackage{graphicx}
\usepackage{array}
\usepackage{float}
\usepackage{wrapfig}
\usepackage{listings}
\usepackage{fixltx2e}
\usepackage{caption}
\usepackage{subcaption}
\usepackage{multirow}
\usepackage[a4paper]{geometry}
\usepackage[runin]{abstract}


\usepackage{color}
\usepackage[usenames,dvipsnames]{xcolor}
\definecolor{bggray}{rgb}{0.95,0.95,0.99}

\usepackage{tikz}
\usepackage{gensymb}

\renewcommand{\abstractname}{Summary: } 
\renewcommand{\absnamepos}{empty}


\usepackage{fancyhdr}
\pagestyle{fancy}
\lhead{\textit{\footnotesize eng-216er}}
\rhead{\textsc{4M17 Optimizing the Bird Function}}
\renewcommand{\headrulewidth}{0pt}


\lstset{language=matlab,
               basicstyle=\color{Blue}\ttfamily\scriptsize,
               keywordstyle=\color{Cerulean}\ttfamily,
               identifierstyle=\color{Black}\ttfamily,
               stringstyle=\color{Fuchsia}\ttfamily,
               commentstyle=\color{Green}\ttfamily,
               backgroundcolor=\color{bggray},
               numberstyle=\footnotesize\color{Gray}\ttfamily,
               numbers=left,
               stepnumber=1,
               frame=leftline,
               rulecolor=\color{Gray},
	       numbersep=10pt,
               breaklines=true,
}



\lstdefinelanguage{JavaScript}{
  keywords={typeof, new, true, false, catch, function, return, null, catch, switch, var, if, in, while, do, else, case, break},
  keywordstyle=\color{blue}\bfseries,
  ndkeywords={class, export, boolean, throw, implements, import, this},
  ndkeywordstyle=\color{darkgray}\bfseries,
  identifierstyle=\color{black},
  sensitive=false,
  comment=[l]{//},
  morecomment=[s]{/*}{*/},
  commentstyle=\color{purple}\ttfamily,
  stringstyle=\color{red}\ttfamily,
  morestring=[b]',
  morestring=[b]"
}

\lstset{language=JavaScript,
               basicstyle=\color{Blue}\ttfamily\scriptsize,
               keywordstyle=\color{Cerulean}\ttfamily,
               identifierstyle=\color{Black}\ttfamily,
               stringstyle=\color{Fuchsia}\ttfamily,
               commentstyle=\color{Green}\ttfamily,
               backgroundcolor=\color{bggray},
               numberstyle=\footnotesize\color{Gray}\ttfamily,
               numbers=left,
               stepnumber=1,
               frame=leftline,
               rulecolor=\color{Gray},
	       numbersep=10pt,
               breaklines=true,
}


\lstset{language=HTML,
               basicstyle=\color{Blue}\ttfamily\scriptsize,
               keywordstyle=\color{Cerulean}\ttfamily,
               identifierstyle=\color{Black}\ttfamily,
               stringstyle=\color{Fuchsia}\ttfamily,
               commentstyle=\color{Green}\ttfamily,
               backgroundcolor=\color{bggray},
               numberstyle=\footnotesize\color{Gray}\ttfamily,
               numbers=left,
               stepnumber=1,
               frame=leftline,
               rulecolor=\color{Gray},
	       numbersep=10pt,
               breaklines=true,
}
\begin{document}

\pagenumbering{roman}

\title{4M17 Exercise \textsc{iii} Optimizing the bird Function}
\author{eng-216er}

\maketitle


\begin{abstract}
Summary goes here.
\end{abstract}

\newpage
\tableofcontents
\listoffigures
\listoftables
\newpage

\pagenumbering{arabic}
\section{ Rationale behind the use of JavaScript }

 JavaScript is unique in that programs written in it can be embedded in a html document, and executed in a web browser. 
No other language can be used for client side web programming without either using a browser extension (Java, Flash) or compiling into JavaScript (CoffeeScript).

This provided the motivation for me to implement the optimisation algorithms in JavaScript. 
In small part, this was because of the possibility of creating a simple html based UI for controlling the optimization parameters and inspecting the results. 

Largely however, I was drawn to using JavaScript because being able to solve optimization problems in a browser could potentially be useful within several web programming contexts. 
For instance, the  development of WebGL allows for hardware accelerated graphics problems to be developed for the web. 
Optimization can be used to solve useful problems in graphics programming.
An example is computing the best possible conformal mapping between texture co-ordinates, and coordinates that make up a mesh of a surface. 
This cam be used to apply a texture to a 3D surface, while minimising the effect of distortion on the surface.

There are currently very few JavaScript optimization libraries. 
Although the software provided in this report does very little to rectify that, it does provide a starting point for more complex software. 


\newpage
\appendix

\section{ Listings }
\subsection{PlotImage.m}

\lstinputlisting[language=matlab]{../PlotImage.m}

\subsection{ Bird.png }

\begin{figure}[H]
\centering
\includegraphics[width=0.5\textwidth]{../Bird.png}
\end{figure}

\subsection{index.html}

\lstinputlisting[language=HTML]{../index.html}

\subsection{minmax.js}

\lstinputlisting[language=JavaScript]{../minmax.js}

\subsection{ex3.js}

\lstinputlisting[language=JavaScript]{../ex3.js}

\subsection{ tabu.js }

\lstinputlisting[language=JavaScript]{../tabu.js}

\subsection{ genetic\_algorithm.js }

\lstinputlisting[language=JavaScript]{../genetic_algorithm.js}

\end{document}
