\documentclass[a4paper,12pt]{article}

\usepackage{a4wide}
\usepackage{amsmath}
\usepackage{graphicx}
\usepackage{array}
\usepackage{float}
\usepackage{wrapfig}
\usepackage{listings}
\usepackage{fixltx2e}
\usepackage{caption}
\usepackage{subcaption}
\usepackage{multirow}
\usepackage[a4paper]{geometry}
\usepackage[runin]{abstract}


\usepackage{color}
\usepackage[usenames,dvipsnames]{xcolor}
\definecolor{bggray}{rgb}{0.95,0.95,0.99}

\usepackage{tikz}
\usepackage{gensymb}

\renewcommand{\abstractname}{Summary: } 
\renewcommand{\absnamepos}{empty}


\usepackage{fancyhdr}
\pagestyle{fancy}
\lhead{\textit{\footnotesize eng-216er}}
\rhead{\textsc{4M17 Optimizing the Bird Function}}
\renewcommand{\headrulewidth}{0pt}


\lstset{language=matlab,
               basicstyle=\color{Blue}\ttfamily\scriptsize,
               keywordstyle=\color{Cerulean}\ttfamily,
               identifierstyle=\color{Black}\ttfamily,
               stringstyle=\color{Fuchsia}\ttfamily,
               commentstyle=\color{Green}\ttfamily,
               backgroundcolor=\color{bggray},
               numberstyle=\footnotesize\color{Gray}\ttfamily,
               numbers=left,
               stepnumber=1,
               frame=leftline,
               rulecolor=\color{Gray},
	       numbersep=10pt,
               breaklines=true,
}



\lstdefinelanguage{JavaScript}{
  keywords={typeof, new, true, false, catch, function, return, null, catch, switch, var, if, in, while, do, else, case, break},
  keywordstyle=\color{blue}\bfseries,
  ndkeywords={class, export, boolean, throw, implements, import, this},
  ndkeywordstyle=\color{darkgray}\bfseries,
  identifierstyle=\color{black},
  sensitive=false,
  comment=[l]{//},
  morecomment=[s]{/*}{*/},
  commentstyle=\color{purple}\ttfamily,
  stringstyle=\color{red}\ttfamily,
  morestring=[b]',
  morestring=[b]"
}

\lstset{language=JavaScript,
               basicstyle=\color{Blue}\ttfamily\scriptsize,
               keywordstyle=\color{Cerulean}\ttfamily,
               identifierstyle=\color{Black}\ttfamily,
               stringstyle=\color{Fuchsia}\ttfamily,
               commentstyle=\color{Green}\ttfamily,
               backgroundcolor=\color{bggray},
               numberstyle=\footnotesize\color{Gray}\ttfamily,
               numbers=left,
               stepnumber=1,
               frame=leftline,
               rulecolor=\color{Gray},
	       numbersep=10pt,
               breaklines=true,
}


\lstset{language=HTML,
               basicstyle=\color{Blue}\ttfamily\scriptsize,
               keywordstyle=\color{Cerulean}\ttfamily,
               identifierstyle=\color{Black}\ttfamily,
               stringstyle=\color{Fuchsia}\ttfamily,
               commentstyle=\color{Green}\ttfamily,
               backgroundcolor=\color{bggray},
               numberstyle=\footnotesize\color{Gray}\ttfamily,
               numbers=left,
               stepnumber=1,
               frame=leftline,
               rulecolor=\color{Gray},
	       numbersep=10pt,
               breaklines=true,
}

\usepackage{url}
\begin{document}

\pagenumbering{roman}

\title{4M17 Exercise III : Optimizing the bird Function}
\author{eng-216er}

\maketitle

\begin{center}
\url{eng-216er.github.io}
\end{center}


\begin{abstract}
Summary goes here.
\end{abstract}

\newpage
\tableofcontents
\listoffigures
\listoftables
\newpage

\pagenumbering{arabic}
\section{ Running the Code }

The code in this report runs as web app.
It can be found in the listings, but is also hosted at \url{eng-216er.github.io} and can be accessed by launching this URL in a web browser. 
The code has been tested in the latest versions of the Firefox and Chrome browsers.  


\section{ Rationale behind the use of JavaScript }

 JavaScript is unique in that programs written in it can be embedded in a html document, and executed in a web browser. 
No other language can be used for client side web programming without either using a browser extension (Java, Flash) or compiling into JavaScript (CoffeeScript).

This provided the motivation for me to implement the optimisation algorithms in JavaScript. 
In small part, this was because of the possibility of creating a simple html based UI for controlling the optimization parameters and inspecting the results. 

Largely however, I was drawn to using JavaScript because being able to solve optimization problems in a browser could potentially be useful within several web programming contexts. 
For instance, the  development of WebGL allows for hardware accelerated graphics problems to be developed for the web. 
Optimization can be used to solve useful problems in graphics programming.
An example is computing the best possible conformal mapping between texture co-ordinates, and coordinates that make up a mesh of a surface. 
This cam be used to apply a texture to a 3D surface, while minimising the effect of distortion on the surface.

There are currently very few JavaScript optimization libraries. 
Although the software provided in this report does very little to rectify that, it does provide a starting point for more complex software. 

\section{Genetic Algorithms}

A genetic algorithm based optimization solver was written for the bird function.
In the genetic algorithm solver, for each individual each parameter x1 and x2 is represented using 16 bits. 
 
Two selection strategies were implemented: tournament selection, and frequency dependant selection. 

Tournament selection was implemented by shuffling the population, and then partitioning the population up into N sets, where N is the number of parents. The best candidate from each set was selected to be a parent.
This guarantees that each candidate may only be selected once. 
It also guarantees that the best candidate will be selected.

Fitness proportionate selection chose N candidates from the population 

The number of parents, and the population size was left configurable. 

\section{Tabu Search}


\newpage
\appendix

\section{ Listings }
\subsection{PlotImage.m}

\lstinputlisting[language=matlab]{../PlotImage.m}

\subsection{ Bird.png }

\begin{figure}[H]
\centering
\includegraphics[width=0.5\textwidth]{../Bird.png}
\end{figure}

\subsection{index.html}

\lstinputlisting[language=HTML]{../index.html}

\subsection{minmax.js}

\lstinputlisting[language=JavaScript]{../minmax.js}

\subsection{ex3.js}

\lstinputlisting[language=JavaScript]{../ex3.js}

\subsection{ tabu.js }

\lstinputlisting[language=JavaScript]{../tabu.js}

\subsection{ genetic\_algorithm.js }

\lstinputlisting[language=JavaScript]{../genetic_algorithm.js}

\end{document}
